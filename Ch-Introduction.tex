Objetivos:
Principal: crear un modelo a partir del uso de la inteligencia artificial que permita facilitar el diagnóstico de la esquizofrenia 

Para cumplir con el objetivo principal se propone:
- Evaluar las conexiones existentes de la corteza prefrontal, analizando las aferencias y eferencias con el resto de la corteza
- Comparar las conexiones previas entre sujetos sanos (grupo 1 control) y sujetos con diagnóstico (grupo 2 experimental)
- Descubrir si alguna de las conexiones encontradas pesa más a la hora de predecir el diagnóstico
\subsection{Distintos abordajes de la esquizofrenia}
La psicosis engloba una serie de síntomas tales como alteraciones significativas en la percepción, los pensamientos, el estado de ánimo y el comportamiento de una persona. Éstos pueden dividirse en síntomas positivos como delirio, alucinaciones y comportamiento, pensamiento y discurso desorganizado como negativos, aplanamiento emocional, discurso reducido, apatìa, abulia, retraìmiento social y autodescuido.(1)
La esquizofrenia (SZ)  es el trastorno psicòtico más frecuente y está asociada con importantes cargas de salud, sociales, ocupacionales y económicas debido a su inicio temprano y a sus síntomas graves y persistentes.(2)
A nivel mundial, la esquizofrenia se encuentra entre las 20 principales causas de discapacidad.(2)
Se considera que esas alteraciones psìquicas se deben a una conectividad anormal a nivel anatómico.(3)
Se evidenció que la alteración en el circuito de ganglios basales-tálamo-córtex desempeña un rol crítico en la fisiopatología se la (SZ). (3)
Dentro de las alteraciones corticales, la afectación de la corteza prefrontal desempeña un rol central en la fisiopatología de la SZ sobre todo en los aspectos cognitivos relacionados con los síntomas negativos.(4) Actualmente el diagnóstico se obtiene fundamentalmente de manera clínica basado en el manual DSM-5, aunque puede apoyarse con tests escalas psicológicas que permiten objetivar los síntomas, sin embargo existen casos en que la información puede ser difícil de obtener por un paciente poco colaborativo, otras alteraciones mentales, pobre cultura sanitaria, dificultades en la lectura, o la afectación propia para reconocer los aspectos inherentes a sì mismo que ocurren en la SZ como también por los sesgos subjetivos de cada profesional de la salud y su interpretación de los manuales diagnósticos.(2, 5) 
Las guías canadienses indican las neuroimágenes únicamente para descartar alguna patología intracraneal como etiología de los síntomas psiquiátricos.(5)
El desarrollo de técnicas diagnósticas para el reconocimiento temprano de alteraciones relacionadas con la esquizofrenia permite el abordaje clìnico precoz tendiente a prevenir las complicaciones de la SZ como la discapacidad, muerte prematura y el suicidio como también un tratamiento dirigido racionalmente y la búsqueda de nuevas dianas terapéuticas.(6)

\subsection{Psiquiatría computacional}

\subsection{Convolutional Neural Networks}
