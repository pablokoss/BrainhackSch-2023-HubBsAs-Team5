Objetivos:
Principal: crear un modelo a partir del uso de la inteligencia artificial que permita facilitar el diagnóstico de la esquizofrenia 

Para cumplir con el objetivo principal se propone:
- Evaluar las conexiones existentes de la corteza prefrontal, analizando las aferencias y eferencias con el resto de la corteza
- Comparar las conexiones previas entre sujetos sanos (grupo 1 control) y sujetos con diagnóstico (grupo 2 experimental)
- Descubrir si alguna de las conexiones encontradas pesa más a la hora de predecir el diagnóstico
\subsection{Distintos abordajes de la esquizofrenia}

\subsection{Psiquiatría computacional}

\subsection{Convolutional Neural Networks}