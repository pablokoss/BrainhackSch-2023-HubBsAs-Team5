\subsection{Base de datos}
Para realizar este trabajo se utilizó un dataset de dominio público obtenido a partir del estudio 
Consortium for Neuropsychiatry Phenomics (CNP), el cual fue fundado por NIH Roadmap Initiative. 
De aquí se tomó una muestra imagénes de resonancia magnética mediante las siguientes técnicas: 
a) MRI ponderadas en T1 con el método MPRAGE ya que permite que se resalten los contrastes entre los tejidos, 
por lo que se utilizó para la segmentación y análisis de la anatomía y 
b) 64 Direction DWI (difusión ponderada en 64 direcciones, se usa para medir ladifusión del agua 
en los tejidos cerebrales). \cite{bilder2020ucla}
\\
Los pacientes analizados tenían entre 21-50 años de edad, todos de Los Ángeles y tenían que cumplir 
con los criterios de ser blancos, no hispanos ni latinos. Previamente se los escaneó para constatar 
que no padecieran enfermedades neurológicas, que no tomaran medicación y que no tengan historial de 
otras enfermedades mentales graves, ADHD ni ningún trastorno de ánimo ni ansiedad. \\
Para este trabajo en específico se eligieron 58 sujetos sanos y 58 sujetos con esquizofrenia diagnosticada.

\subsection{Preprocesamiento y procesamiento de imágenes}
Una vez escogidas las imágenes se realizó un preprocesamiento mediante el software FSL con el objetivo de 
corregir movimientos, distorsión, eliminar ruidos y segmentar tejidos para poder así separar las regiones 
de interés (ROI) y mejorar la precisión de las tractografías y mediciones.\\
Luego se procedió a procesar las imágenes mediante el programa DSI Studio para extraer características 
tractgráficas. Puntualmente se evaluó la longitud, la fracción de anisotropía, la curvatura y el volumen 
de los tractos. \\
Todos estos datos obtenidos se utilizaron como inputs para el análisis de las imágenes mediante redes 
neuronales convolucionales.  

\subsection{Entrenamiento de la CNN}
Primero se definió la arquitectura de la red neuronal con X cantidad de capas convolucionales, X capas de 
agrupación, X capas de normalización. Posterior a esto se compiló la CNN para definir la función de pérdida, 
el optimizador y las métricas de evaluación. Una vez definido todo esto se procedió a entrenar a la CNN: 
se ingresaron los inputs. \\
Se utilizó una fórmula de regresión logística para definir cada nodo (1). El nodo (y) es el resultado de 
multiplicar el peso (w) por un set de variables independientes (x), a eso se le sumó un bias. Luego a ese 
resultado se le aplicó la función de activación (f).
\begin{equation}
    Y = f (wX+ b)
\end{equation}
Al principio se usaron valores random de w y b y se procedió a repetir el procedimiento para minimizar los 
errores de predicción de los valores de inputs  outputs ya conocidos. Se repitió esto hasta reducir el error 
a un valor aceptable (p<0,05)\\
Luego se procedió a mejorar el modelo agregándole más capas (nº de capas)y nodos (nº de nodos).\\

\subsection{Análisis estadístico}
Cuando se completó el entrenamiento del modelo se procedió a realizar el análisis estadístico mediante una 
prueba t de Student para comparar la diferencia entre el total de imágenes y el total de aciertos del modelo.